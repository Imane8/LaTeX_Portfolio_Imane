\documentclass[12pt,a4paper,twoside,openright]{report}
% ======================================================================
% PARTIE 1 : CONFIGURATION DE BASE DU DOCUMENT
% ======================================================================
\renewcommand{\thechapter}{\Roman{chapter}}

% ======================================================================
% PARTIE 2 : PAQUETS ESSENTIELS (ORDRE OPTIMAL)
% ======================================================================
\usepackage[utf8]{inputenc}                % Encodage des caractères
\usepackage[T1]{fontenc}                   % Meilleur support des accents
\usepackage[french]{babel}                 % Langue française
\usepackage{aeguill}                       % Guillemets français
\usepackage{ae}                            % Font extensions

% ======================================================================
% PARTIE 3 : PAQUETS DE MISE EN PAGE ET GRAPHIQUES
% ======================================================================
\usepackage[width=170mm,top=25mm,bottom=20mm]{geometry} % Marges
\usepackage{setspace}                      % Interligne
\usepackage{fancyhdr}                      % En-têtes et pieds de page
\usepackage{titlesec}                      % Formatage des titres
\usepackage{eso-pic}                       % Positionnement absolu
\usepackage{xspace}                        % Espacement intelligent
\usepackage{float}                         % Positionnement des flottants
\usepackage{rotating}                      % Rotation d'éléments
\usepackage[center]{subfigure}             % Sous-figures
\usepackage{longtable}                     % Tableaux longs
\usepackage{multirow}                      % Cellules fusionnées
\usepackage{array}                         % Tableaux avancés
\usepackage{enumitem}                      % Listes personnalisées

\usepackage{algorithm}      % Pour les environnements algorithm
\usepackage{algpseudocode}  % Pour le pseudocode (style algorithmique)

% ======================================================================
% PARTIE 4 : PAQUETS MATHEMATIQUES ET SCIENTIFIQUES
% ======================================================================
\usepackage{amsmath,amssymb}               % Mathématiques
\usepackage{siunitx}                       % Unités scientifiques
\usepackage{breakcites}                    % Césure des citations

% ======================================================================
% PARTIE 5 : PAQUETS VISUELS ET COULEURS
% ======================================================================
\usepackage{graphicx}                      % Images (version standard)
\usepackage{epsfig}                        % Figures EPS (obsolète, mais conservé)
\usepackage{color,soul}                    % Couleurs et surlignage
\usepackage[font={it}]{caption}           % Légendes en italique     

% ======================================================================
% PARTIE 6 : PAQUETS SPECIFIQUES AU PROJET
% ======================================================================
\usepackage[french]{minitoc}               % Mini tables des matières
\usepackage{frbib_jlh}                     % Style bibliographique

% ======================================================================
% PARTIE 7 : CONFIGURATIONS GLOBALES
% ======================================================================
\setcounter{minitocdepth}{5}               % Profondeur de la mini-TdM
\setcounter{secnumdepth}{3}                % Profondeur de numérotation
\setcounter{tocdepth}{3}                   % Profondeur de la TdM

% ======================================================================
% PARTIE 8 : CONFIGURATION HYPERREF (DOIT ÊTRE APRÈS LA PLUPART DES PAQUETS)
% ======================================================================

\usepackage[bookmarks=true,bookmarksopen=true]{hyperref}

\hypersetup{
    pdftitle={Mémoire de thèse Tuan-Long CHU},
    pdfauthor={Tuan-Long CHU},
    pdfsubject={Le sujet de ce mémoire de thèse porte sur...},
    colorlinks=true,
    linkcolor=black,
    citecolor=black,
    urlcolor=black
}

% ======================================================================
% PARTIE 9 : PERSONNALISATION DES EN-TÊTES ET PIEDS DE PAGE
% ======================================================================
\pagestyle{fancy}
\newcommand{\changefont}{\fontsize{12}{14}\selectfont}
\renewcommand{\chaptermark}[1]{\markboth{\thechapter\ #1}{\thechapter\ #1}}
\renewcommand{\sectionmark}[1]{\markright{#1}}
\fancyhf{}

% Style pour les pages "plain" (début de chapitre)
\fancypagestyle{plain}{%
    \fancyhf{}%
    \fancyfoot[C]{- \thepage\ -}%
    \renewcommand{\headrulewidth}{0pt}% Pas de ligne d'en-tête
    \renewcommand{\footrulewidth}{0.0pt}% Ligne de pied de page invisible
}

% Configuration par défaut
\fancyfoot[C]{\em\thepage}
\fancyhead[LO]{\em\leftmark}
\fancyhead[RE]{\em\leftmark}

% Augmenter la hauteur de l'en-tête pour éviter les warnings
\setlength{\headheight}{14.5pt}

% ======================================================================
% PARTIE 10 : ABRÉVIATIONS MATHÉMATIQUES
% ======================================================================
\let\d=\partial
\let\f=\frac
\let\lam=\lambda
\let\eps=\varepsilon
\let\vect=\overrightarrow
\let\bs=\bigskip
\let\wt=\widetilde
\let\ol=\overline
\let\ds=\displaystyle

% ======================================================================
% PARTIE 11 : DÉFINITIONS FRANÇAISES
% ======================================================================
\addto\captionsfrench{
    \def\figurename{{\itshape Figure}}
    \def\tablename{{\itshape Tableau}}
    \def\listfigurename{Liste des figures}
    \def\listtablename{Liste des tableaux}
}

% ======================================================================
% PARTIE 12 : ENVIRONNEMENTS PERSONNALISÉS
% ======================================================================
\newenvironment{changemargin}[2]{%
    \begin{list}{}{%
        \setlength{\topsep}{0pt}%
        \setlength{\topmargin}{#1}%
        \setlength{\bottommargin}{#2}%
        \setlength{\listparindent}{\parindent}%
        \setlength{\itemindent}{\parindent}%
        \setlength{\parsep}{\parskip}%
    }%
    \item[]}{\end{list}}


% ======================================================================
% PARTIE 13 : DÉBUT DU DOCUMENT
% ======================================================================
\begin{document}
% ----------------------------------------------------------------------
% PAGES PRÉLIMINAIRES (SANS NUMÉROTATION OU NUMÉROTATION ROMAINE)
% ----------------------------------------------------------------------
\pagestyle{empty} % Pas d'en-têtes pour les pages préliminaires

% Page de garde
\input{./page_de_garde/page_de_garde}
\cleardoublepage

% Remerciements
\input{./remerciements/remerciements}
\clearpage

% Résumé
\input{./resume/resume}
\clearpage

% Dédicaces
\input{./Dedicaces/Dedicaces}
\clearpage

% Liste des symboles et abréviations
\input{./Symboles/Symboles}
\clearpage

% ----------------------------------------------------------------------
% TABLES DES MATIÈRES ET LISTES
% ----------------------------------------------------------------------
\pagestyle{fancy}
\fancyhead[LO]{\changefont \em Table des matières}
\fancyhead[RE]{\changefont \em Table des matières}
\fancyfoot[C]{- \thepage \ -}
\pagenumbering{Roman} % Numérotation en chiffres romains

% Table des matières principale
\pdfbookmark[0]{Table des matières}{Table des matières}
\tableofcontents
\cleardoublepage

% Liste des figures
\fancyhead[LO]{\changefont \em Liste des figures}
\fancyhead[RE]{\changefont \em Liste des figures}
\pdfbookmark[0]{Liste des figures}{liste des figures}
\listoffigures
\cleardoublepage

% Liste des tableaux
\fancyhead[LO]{\changefont \em Liste des tableaux}
\fancyhead[RE]{\changefont \em Liste des tableaux}
\pdfbookmark[0]{Liste des tableaux}{liste des tableaux}
\listoftables
\cleardoublepage

% ----------------------------------------------------------------------
% CORPS DU DOCUMENT (NUMÉROTATION ARABE)
% ----------------------------------------------------------------------
\begin{onehalfspacing} % Interligne 1.5

% Introduction générale
\addcontentsline{toc}{chapter}{Introduction générale}
\fancyhead[LO]{\changefont \em Introduction générale}
\fancyhead[RE]{\changefont \em Introduction générale}
\fancyfoot[CE,CO]{- \thepage \ -}
\include{./introduction/introduction}

% Chapitre 1
\cleardoublepage
\fancyhf{}
\fancyfoot[CE,CO]{- \thepage \ -}
\renewcommand{\changefont}{\fontsize{9}{12}\selectfont}
\fancyhead[LO]{\changefont \textit{Chapitre} \em\leftmark}
\fancyhead[RE]{\changefont \textit{Chapitre} \em\rightmark}
\input{./chapitre_1/chapitre_1}

% Chapitre 2
\cleardoublepage
\fancyhf{}
\fancyfoot[CE,CO]{- \thepage \ -}
\renewcommand{\changefont}{\fontsize{9}{12}\selectfont}
\fancyhead[LO]{\changefont \textit{Chapitre} \em\leftmark}
\fancyhead[RE]{\changefont \textit{Chapitre} \em\rightmark}
\input{./chapitre_2/chapitre_2}

% Chapitre 3

\fancyhf{}
\fancyfoot[CE,CO]{- \thepage \ -}
\renewcommand{\changefont}{\fontsize{9}{12}\selectfont}
\fancyhead[LO]{\changefont \textit{Chapitre} \em\leftmark}
\fancyhead[RE]{\changefont \textit{Chapitre} \em\rightmark}
\input{./chapitre_3/chapitre_3}

% Conclusion générale
\clearpage
\fancyhead[LO]{\em Conclusions générales et perspectives}
\fancyhead[RE]{\em Conclusions générales et perspectives}
\fancyfoot[C]{- \thepage \ -}
\include{./conclusion/conclusion}

\end{onehalfspacing}

% ----------------------------------------------------------------------
% RÉFÉRENCES BIBLIOGRAPHIQUES
% ----------------------------------------------------------------------

\addcontentsline{toc}{chapter}{Références bibliographiques}
\fancyhead[LO]{\em Références bibliographiques}
\fancyhead[RE]{\em Références bibliographiques}
\fancyfoot[C]{- \thepage \ -}

\bibliographystyle{plain}

\begin{thebibliography}{99}

	\bibitem{Ali} Ali Mustafa Qamar, Eric Gaussier Apprentissage de différentes classes de similarité dans les K-PPV 2016.
	
	 \bibitem{Baggin} Introduction aux méthodes d'agrégation : boosting, baggin et forêts aléatoires. Illustrations avec R. université Rennes 2.
	 
	 \bibitem{Baïna} Baïna S., Panetto H. et Benali K., Apport de l'approche MDA pour une interopérabilité sémantique : Interopérabilité des systèmes d'information d'entreprise, Processus d?entreprise et SI, RSTI-ISI, pp.11-29, novembre 2006.
	 
	 \bibitem{BARIGOU} BARIGOU Fatiha CONTRIBUTION À LA CATÉGORISATION DE TEXTES ET À L'EXTRACTION D'INFORMATION.
	 	
	 \bibitem{Benchettouh} BENCHETTOUH Salah Eddine  Elaboration d'un système de prédiction des pannes et de planification des maintenances.
	 
	\bibitem{BREIMAN} Leo Breiman, Random Forests 1999-2001.
	
	\bibitem{Bruno} Bruno Taconet, Abderrazak Zahour, Saîd Ramadane, Wafa Boussella Cllassification des K-PPV par sous-voisinages embloîtés 2006.		
	 		
	\bibitem{Boucly} Boucly F., Le management de la maintenance : Evolution et mutation, Editions Afnor, 1998.
	 
	\bibitem{Chapman} Chapman et Hall DATA Classification Algorithms and Applications.
	
	\bibitem{Christophe} Christophe Chesneau éléments de classification	2016.
	    
	\bibitem{Cornuéjols} A.Cornuéjols, L.Miclet, Y.Kodratoff Apprentissage artificiel EYROLLES.
	
	\bibitem{DJEFFAL} Abdelhamid DJEFFAL.Utilisation des méthodes Support Vector Machine (SVM) dans l'analyse des bases de données Université Mohamed Khider, Biskra Abdelhamid DJEFFAL.
		
	\bibitem{Eric} Ali Mustafa Qamar, Eric Gaussier Similarity learning in Nearest Neighbor and Application to information Retrieval 2014.
			
	\bibitem{Ethem} Ethem Alpaydin Introduction to machine learning.
	
	\bibitem{Eve} Eve Mathieu-Dupas Algorithme des k plus proches voisins pondérés et application en diagnostic 2010.
		
		
	\bibitem{Fabien} Fabien Torre un algorithme stochastique pour l'apprentissage supervisé et non-supervisé.
	
	\bibitem{Faïcel} Faïcel Chamroukhi.Classification supervisé : les K-plus proches voisins, université de Caen.
	
	\bibitem{Faouzi} M. ZAIZ Faouzi  Les Supports Vecteurs Machines (SVM) pour la reconnaissance des caractères manuscrits arabes université Mohamed Khider Biskra
		
	\bibitem{Francastel} Francastel J.C., Externalisation de la maintenance : Stratégies, méthodes et contrats. Dunod, Paris, 2003.
	
	\bibitem{Intro} K.Gosalia,Ph.D., CFA, CPA, CGA Rock Lefebvre, MBA, FCIS, FCPA, FCGA Introduction à l'apprentissage automatique; MONOGRAPHIE DE CPA NOUVEAU-BRUNSWICK; 
		
		
	\bibitem{Kaffel} Kaffel H., La maintenance distribuée: concept, évaluation et mise en oeuvre. Thèse de doctorat, Université Laval,
	Quebec, 2001.
	
	\bibitem{Kahn} Kahn J., Overview of MIMOSA and the Open System Architecture for Enterprise Application Integration. Proc.
	of COMADEM 2003, pp. 661-670, Växjö University, Sweden, 2003.
	
	
	\bibitem{Kamalesh} kamalesh Gosalia, Roch Lefebvre Introduction à l'apprentissage automatique.
		
	\bibitem{Karem} F.Karem, M.Dhibi, A.Martin Combinaison de classification supervisée et non-supervisée par la théorie des fonctions de croyance.
		
	\bibitem{Loughani} A.LOUGHANI + JNB cours.
			
	\bibitem{Machine} Introduction to Machine Learning with Python Andreas C. Müller Sarah Guido.
	
	\bibitem{Norman} Norman Matloff Statistical Regression and Classification.	    
		    
	\bibitem{Mifdal} Rachid MIFDAL  PReSENTe a L'eCOLE DE TECHNOLOGIE SUPeRIEURE COMME EXIGENCE PARTIELLE À L'OBTENTION DE LA MAÎTRISE AVEC MeMOIRE EN GeNIE, CONCENTRATION PERSONNALISeE M. Sc. A.
		
	\bibitem{Mce} F.Monchy, J.Vernier MAINTENANCE Méthodes et organisations DUNOD.
	
 	\bibitem{Rudolph} Rudolph Russell Machine learning.
 	
	\bibitem{Zerhouni} I.Rasovska, B.Chebel-Morello, N.Zerhouni.  Classification des différentes architectures en maintenance. 
	
\end{thebibliography}
	

% ----------------------------------------------------------------------
% ANNEXES
% ----------------------------------------------------------------------
\cleardoublepage
\appendix % Commande pour indiquer le début des annexes
\pagestyle{fancy}
\fancyhead[LO]{\changefont \em ANNEXES}
\fancyhead[RE]{\changefont \em ANNEXES}
\fancyfoot[CE,CO]{- \thepage \ -}
\include{./ANNEXES/annexes}

\end{document}